
%----------------------------------------------------------------------------------------
%	PACKAGES AND OTHER DOCUMENT CONFIGURATIONS
%----------------------------------------------------------------------------------------

\documentclass[paper=a4, fontsize=11pt]{scrartcl} % A4 paper and 11pt font size

\usepackage[T1]{fontenc} % Use 8-bit encoding that has 256 glyphs
\usepackage{graphicx}
\usepackage{fourier} % Use the Adobe Utopia font for the document - comment this line to return to the LaTeX default
\usepackage[english]{babel} % English language/hyphenation
\usepackage{amsmath,amsfonts,amsthm} % Math packages

\usepackage{lipsum} % Used for inserting dummy 'Lorem ipsum' text into the template

\usepackage{sectsty} % Allows customizing section commands
\allsectionsfont{\centering \normalfont\scshape} % Make all sections centered, the default font and small caps

\usepackage{fancyhdr} % Custom headers and footers
\pagestyle{fancyplain} % Makes all pages in the document conform to the custom headers and footers
\fancyhead{} % No page header - if you want one, create it in the same way as the footers below
\fancyfoot[L]{} % Empty left footer
\fancyfoot[C]{} % Empty center footer
\fancyfoot[R]{\thepage} % Page numbering for right footer
\renewcommand{\headrulewidth}{0pt} % Remove header underlines
\renewcommand{\footrulewidth}{0pt} % Remove footer underlines
\setlength{\headheight}{13.6pt} % Customize the height of the header

\numberwithin{equation}{section} % Number equations within sections (i.e. 1.1, 1.2, 2.1, 2.2 instead of 1, 2, 3, 4)
\numberwithin{figure}{section} % Number figures within sections (i.e. 1.1, 1.2, 2.1, 2.2 instead of 1, 2, 3, 4)
\numberwithin{table}{section} % Number tables within sections (i.e. 1.1, 1.2, 2.1, 2.2 instead of 1, 2, 3, 4)

\setlength\parindent{0pt} % Removes all indentation from paragraphs - comment this line for an assignment with lots of text

%----------------------------------------------------------------------------------------
%	TITLE SECTION
%----------------------------------------------------------------------------------------

\newcommand{\horrule}[1]{\rule{\linewidth}{#1}} % Create horizontal rule command with 1 argument of height

\title{	
\normalfont \normalsize 
\textsc{EEP702, Software Laboratory} \\ [25pt] % Your university, school and/or department name(s)
\horrule{0.5pt} \\[0.4cm] % Thin top horizontal rule
\huge Java Programming Assignment \\ % The assignment title
\horrule{2pt} \\[0.5cm] % Thick bottom horizontal rule
}

\author{Rishi Dua, 2010EE50557} % Your name

\date{\normalsize 12 January, 2014} % Today's date or a custom date

\begin{document}

\maketitle % Print the title

%----------------------------------------------------------------------------------------
%	PROBLEM 1
%----------------------------------------------------------------------------------------

\section{Reverse Spiral}
Write a program in C/C++/Java to solve following two problems:

\subsection{Problem statement}
To print matrix (M) elements in reverse spiral order i.e., starting from the centre element, print all the elements in spiral order until the first element M[0][0] is reached. Matrix M is of order n where n is odd

\subsection{Abstract}

Traversal (also known as tree search) refers to the process of visiting (examining and/or updating) each node in a tree data structure, exactly once, in a systematic way. The problem aims at traversing a matrix in reverse spiral order.\\

The java code aims at traversing through all elements of a matrix in a reverse spiral form. The problem involves using loops and recursion to match two strings containing wildcard characters '?' and '*' . '?' denotes no or exactly one character and '*' denotes no or many characters.

\subsection{Specification And Assumptions}
{\textbf {Specifications:}}\\
Language used: Java\\
Platform: Ubuntu 12.04\\
Libraries used: java.io.BufferedReader\\
IDE: Eclipse\\

{\textbf {Assumptions:}}\\
n is odd

\newpage
\subsection{Flow chart}
 {\center\includegraphics[height=10 in]{assgn1part1.png}}

\subsection{Logic Implementation}
Start from the centremost element.\\
Move in the following order: left, down, right and top increasing the number of steps moved by one every time.\\

\subsection{Execution Directive}
java ReverseSpiral
\subsection{Output Of The Program}
Enter value for n:\\
\\
5\\
5 by 5 matrix\\
\\
Enter values for following:\\
\\
(0,0)\\
1\\
(0,1)\\
2\\
(0,2)\\
3\\
(0,3)\\
4\\
(0,4)\\
5\\
(1,0)\\
6\\
(1,1)\\
7\\
(1,2)\\
8\\
(1,3)\\
9\\
(1,4)\\
10\\
(2,0)\\
11\\
(2,1)\\
12\\
(2,2)\\
13\\
(2,3)\\
14\\
(2,4)\\
15\\
(3,0)\\
16\\
(3,1)\\
17\\
(3,2)\\
18\\
(3,3)\\
19\\
(3,4)\\
20\\
(4,0)\\
21\\
(4,1)\\
22\\
(4,2)\\
23\\
(4,3)\\
24\\
(4,4)\\
25\\
Output:\\
13\\
12\\
17\\
18\\
19\\
14\\
9\\
8\\
7\\
6\\
11\\
16\\
21\\
22\\
23\\
24\\
25\\
20\\
15\\
10\\
5\\
4\\
3\\
2\\
1\\


\subsection{Result}
The program prints matrix (M) elements in reverse spiral order i.e., starting from the centre element, print all the elements in spiral order until the first element M[0][0] is reached. Matrix M is of order n where n is odd

\subsection{Conclusion}
Successfully traversed through the elements of a matrix in reverse-spiral order as required.

\section{String Matching}

\subsection{Problem statement}
To match two strings containing wildcard characters '?' and '*' . '?' denotes no or exactly one character and '*' denotes no or many characters.

\subsection{Abstract}
A regular expression (abbreviated regex or regexp) is a sequence of characters that forms a search pattern, mainly for use in pattern matching with strings. The problem aims at matching two strings containing wildcards.\\

The problem is solved using Java. The problem involves using loops and recursion to match two strings containing wildcard characters '?' and '*' . '?' denotes no or exactly one character and '*' denotes no or many characters.

\subsection{Specification And Assumptions}
{\textbf {Specifications:}}\\
Language used: Java\\
Platform: Ubuntu 12.04\\
Libraries used: java.io.BufferedReader\\
IDE: Eclipse\\

{\textbf {Assumptions:}}\\
Only * and ? are treated as wildcards

\newpage
\subsection{Flow chart}
 {\center\includegraphics[height=10 in]{assgn1part1.png}}

\subsection{Logic Implementation}
Recursive calling is used\\
First of all, trivial cases are checked\\
If the first character is a ?, the remaining string is matched with the other string as it is and also with the first character removed
If the first character is a ?, the remaining string is matched with the other string as it is and also with all the permutations beginning with the next non-wildcard character of the first string


\subsection{Execution Directive}
java RegexMatcher

\subsection{Output Of The Program}
Enter string 1\\
hello\\
Enter string 2\\
h*llo?\\
result: the strings match\\

\subsection{Result}
The program can do a simple regex comparison of two given strings.\\
Normal regex comparison works when both the strings contain wildcards. This code works even then.

\subsection{Conclusion}
Successfully compared two strings, when one/both/none contain ? and * wildcard characters.

\end{document}