
%----------------------------------------------------------------------------------------
%	PACKAGES AND OTHER DOCUMENT CONFIGURATIONS
%----------------------------------------------------------------------------------------

\documentclass[paper=a4, fontsize=11pt]{scrartcl} % A4 paper and 11pt font size

\usepackage[T1]{fontenc} % Use 8-bit encoding that has 256 glyphs
\usepackage{graphicx}
\usepackage{fourier} % Use the Adobe Utopia font for the document - comment this line to return to the LaTeX default
\usepackage[english]{babel} % English language/hyphenation
\usepackage{amsmath,amsfonts,amsthm} % Math packages

\usepackage{lipsum} % Used for inserting dummy 'Lorem ipsum' text into the template

\usepackage{sectsty} % Allows customizing section commands
\allsectionsfont{\centering \normalfont\scshape} % Make all sections centered, the default font and small caps

\usepackage{fancyhdr} % Custom headers and footers
\pagestyle{fancyplain} % Makes all pages in the document conform to the custom headers and footers
\fancyhead{} % No page header - if you want one, create it in the same way as the footers below
\fancyfoot[L]{} % Empty left footer
\fancyfoot[C]{} % Empty center footer
\fancyfoot[R]{\thepage} % Page numbering for right footer
\renewcommand{\headrulewidth}{0pt} % Remove header underlines
\renewcommand{\footrulewidth}{0pt} % Remove footer underlines
\setlength{\headheight}{13.6pt} % Customize the height of the header

\numberwithin{equation}{section} % Number equations within sections (i.e. 1.1, 1.2, 2.1, 2.2 instead of 1, 2, 3, 4)
\numberwithin{figure}{section} % Number figures within sections (i.e. 1.1, 1.2, 2.1, 2.2 instead of 1, 2, 3, 4)
\numberwithin{table}{section} % Number tables within sections (i.e. 1.1, 1.2, 2.1, 2.2 instead of 1, 2, 3, 4)

\setlength\parindent{0pt} % Removes all indentation from paragraphs - comment this line for an assignment with lots of text

%----------------------------------------------------------------------------------------
%	TITLE SECTION
%----------------------------------------------------------------------------------------

\newcommand{\horrule}[1]{\rule{\linewidth}{#1}} % Create horizontal rule command with 1 argument of height

\title{	
\normalfont \normalsize 
\textsc{EEP702, Software Laboratory} \\ [25pt] % Your university, school and/or department name(s)
\horrule{0.5pt} \\[0.4cm] % Thin top horizontal rule
\huge Library Management \\ % The assignment title
\horrule{2pt} \\[0.5cm] % Thick bottom horizontal rule
}

\author{Rishi Dua, 2010EE50557} % Your name

\date{\normalsize 28 January, 2014} % Today's date or a custom date

\begin{document}

\maketitle % Print the title

%----------------------------------------------------------------------------------------
%	PROBLEM 1
%----------------------------------------------------------------------------------------

\section{Library Management}


\subsection{Problem statement}
Write a program in Java only for Library Management.
\begin{enumerate}
\item Use a text file to store the following information about the books available in the library.
(Title of the book) (Author name) (Publication) (Edition)
Read from the file and store them as array of objects. Provide user the functionality to 
search for some books based on Book title, author name, publication or any combination of 
these.
\item Add the Library Incharge Login functionality to above program which enables him to add 
or remove any book from the library. The changes must be reflected in the Library record 
and all student accounts as well.
\item Add the Student Login functionality to above program to keep a record of the books 
issued to each student along with the date of issue.
\item Add the functionality to above program to calculate total fine imposed in case any user 
fails to deposit the issued book within a period of 1 week.
\end{enumerate}

All objective optional functions are implemented.

\subsection{Abstract}

In the real world, we often find many individual objects all of the same kind. There may be thousands of other bicycles in existence, all of the same make and model. Each book was built from the same set of blueprints and therefore contains the same components. In object-oriented terms, we say that your book is an instance of the class of objects known as books. A class is the blueprint from which individual objects are create\\

\subsection{Specification And Assumptions}
{\textbf {Tool Specifications:}}\\
Language used: Java\\
Platform: Ubuntu 12.04\\
Additional tools used: none\\
Eclipse Version: Version: 3.7.2\\

{\textbf{Assumptions}}\\
The input file already exists in the specified TSV format\\

\textbf{Problem specifications}\\
This is the complete list of members for LibManagement, including all inherited members.\\

addbook()	LibManagement	 [static]\\
booklist (defined in LibManagement)	LibManagement	 [package, static]\\
displaybooks()	LibManagement	 [static]\\
i	LibManagement	 [package, static]\\
loggedin (defined in LibManagement)	LibManagement	 [package, static]\\
main(String[] args) (defined in LibManagement)	LibManagement	 [static]\\
makelist()	LibManagement	 [static]\\
removebook()	LibManagement	 [static]\\
searchbooks()	LibManagement	 [static]\\

\textbf{Class book}\\


Public Member Functions\\
\\
 	Book (String[] parts)\\
String 	gettitle ()\\
String 	getauthor ()\\
String 	getpuiblication ()\\
String 	getedition ()\\
Package Attributes\\
\\
String 	title\\
String 	author\\
String 	publication\\
String 	edition\\
Detailed Description\\
\\

Constructor \& Destructor Documentation\\
Book.Book	(	String[] 	parts	)	\\
Parameters:\\
parts	\\

\newpage
\subsection{Flow chart}
 {\center\includegraphics[height=4in]{flowchart.jpg}}


\subsection{Logic Implementation}
The problem is broken into the following 7 parts:
1: login\\
2: view all\\
3: search\\
4: add (admin only)\\
5: remove (admin only)\\
6: logout\\
7::exit\\

\subsection{Execution Directive}
\textbf{Compiling}:
javac LibManagement.java

\textbf{Running}:\\
java LibManagement\\

Follow the on-screen instructions after that

\subsection{Output Of The Program}
\begin{verbatim}
Library Manager
Using as Guest
Enter choice: 1: login, 2: view all, 3: search 4: add (admin only), 5: remove (admin only), 6: logout, 7:exit
2
The list of books is as follows:

A	D	B	C
E	H	F	G
I	L	J	K
Tit	Ed	Aut	Pub
qwe	rtytit1	wer	ert
au2	random	pub2	ed2
au	dua	pi	edrishi
www	www	www	www
random	random	random	random
newnew	newnew	newnew	newnew
rishi	ed	dua	pub
Using as Guest
Enter choice: 1: login, 2: view all, 3: search 4: add (admin only), 5: remove (admin only), 6: logout, 7:exit
3
Enter string to search
D
The book is A, ed: D by B. Publication: C
Using as Guest
Enter choice: 1: login, 2: view all, 3: search 4: add (admin only), 5: remove (admin only), 6: logout, 7:exit
2
The list of books is as follows:

A	D	B	C
E	H	F	G
I	L	J	K
Tit	Ed	Aut	Pub
qwe	rtytit1	wer	ert
au2	random	pub2	ed2
au	dua	pi	edrishi
www	www	www	www
random	random	random	random
newnew	newnew	newnew	newnew
rishi	ed	dua	pub
Using as Guest
Enter choice: 1: login, 2: view all, 3: search 4: add (admin only), 5: remove (admin only), 6: logout, 7:exit
1
Enter username
admin
Enter username
pass
Hello admin!
Logged in as admin
Enter choice: 1: login, 2: view all, 3: search 4: add (admin only), 5: remove (admin only), 6: logout, 7:exit
5
The list of books is as follows:

ID	0	A, ed: D by B. Publication: C
ID	1	E, ed: H by F. Publication: G
ID	2	I, ed: L by J. Publication: K
ID	3	Tit, ed: Ed by Aut. Publication: Pub
ID	4	qwe, ed: rtytit1 by wer. Publication: ert
ID	5	au2, ed: random by pub2. Publication: ed2
ID	6	au, ed: dua by pi. Publication: edrishi
ID	7	www, ed: www by www. Publication: www
ID	8	random, ed: random by random. Publication: random
ID	9	newnew, ed: newnew by newnew. Publication: newnew
ID	10	rishi, ed: ed by dua. Publication: pub
Enter ID of book to delete
7
Logged in as admin
Enter choice: 1: login, 2: view all, 3: search 4: add (admin only), 5: remove (admin only), 6: logout, 7:exit
7

\end{verbatim}
\subsection{Result}
A Libray manager is developed\\
The code is developed with Java\\

The user is able to add/remove/view all books.\\
the user can login as admin. This gives him write privilidges to the database.\\

The documentation made using doxygen is also attached with the submission\\


\subsection{Conclusion}
Successfully developed a library manager.
Key things learnt\\
Authentication is the act of confirming the truth of an attribute of a datum or entity. This might involve confirming the identity of a person or software program, tracing the origins of an artifact, or ensuring that a product is what its packaging and labeling claims to be. Authentication often involves verifying the validity of at least one form of identification. This is used for login.\\
Reading and writing files from java using buffered readers is learnt. The details of reading, writing, creating, and opening files. There are a wide array of file I/O methods to choose from which were explored in this  project.\\
\end{document}
