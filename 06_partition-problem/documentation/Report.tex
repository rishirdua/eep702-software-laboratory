
%----------------------------------------------------------------------------------------
%	PACKAGES AND OTHER DOCUMENT CONFIGURATIONS
%----------------------------------------------------------------------------------------

\documentclass[paper=a4, fontsize=11pt]{scrartcl} % A4 paper and 11pt font size

\usepackage[T1]{fontenc} % Use 8-bit encoding that has 256 glyphs
\usepackage{graphicx}
\usepackage{fourier} % Use the Adobe Utopia font for the document - comment this line to return to the LaTeX default
\usepackage[english]{babel} % English language/hyphenation
\usepackage{amsmath,amsfonts,amsthm} % Math packages

\usepackage{lipsum} % Used for inserting dummy 'Lorem ipsum' text into the template

\usepackage{sectsty} % Allows customizing section commands
\allsectionsfont{\centering \normalfont\scshape} % Make all sections centered, the default font and small caps

\usepackage{fancyhdr} % Custom headers and footers
\pagestyle{fancyplain} % Makes all pages in the document conform to the custom headers and footers
\fancyhead{} % No page header - if you want one, create it in the same way as the footers below
\fancyfoot[L]{} % Empty left footer
\fancyfoot[C]{} % Empty center footer
\fancyfoot[R]{\thepage} % Page numbering for right footer
\renewcommand{\headrulewidth}{0pt} % Remove header underlines
\renewcommand{\footrulewidth}{0pt} % Remove footer underlines
\setlength{\headheight}{13.6pt} % Customize the height of the header

\numberwithin{equation}{section} % Number equations within sections (i.e. 1.1, 1.2, 2.1, 2.2 instead of 1, 2, 3, 4)
\numberwithin{figure}{section} % Number figures within sections (i.e. 1.1, 1.2, 2.1, 2.2 instead of 1, 2, 3, 4)
\numberwithin{table}{section} % Number tables within sections (i.e. 1.1, 1.2, 2.1, 2.2 instead of 1, 2, 3, 4)

\setlength\parindent{0pt} % Removes all indentation from paragraphs - comment this line for an assignment with lots of text

%----------------------------------------------------------------------------------------
%	TITLE SECTION
%----------------------------------------------------------------------------------------

\newcommand{\horrule}[1]{\rule{\linewidth}{#1}} % Create horizontal rule command with 1 argument of height

\title{	
\normalfont \normalsize 
\textsc{EEP702, Software Laboratory} \\ [25pt] % Your university, school and/or department name(s)
\horrule{0.5pt} \\[0.4cm] % Thin top horizontal rule
\huge Partition Problem \\ % The assignment title
\horrule{2pt} \\[0.5cm] % Thick bottom horizontal rule
}

\author{Rishi Dua, 2010EE50557} % Your name

\date{\normalsize 9 February, 2014} % Today's date or a custom date

\begin{document}

\maketitle % Print the title

%----------------------------------------------------------------------------------------
%	PROBLEM 1
%----------------------------------------------------------------------------------------

\section{Partition Problem}


\subsection{Problem statement}

There are N packets, each with one or more candies. There are K students among which the packets have to be distributed. (Assume K less N for all cases). The parameters N and K have to be provided by the user at run-time. Each student gets only one packet. The number of candies in various packets are (x1, x2, x3,....xk ) , where xi denotes the number of candies in the ith packet. Find the number of triplets (x1, x2, x3) possible such that sum of the candies (x1 + x2 + x3) is even.\\
Divide the packets into two parts (p1 and p2) such that the difference (|p1-p2|) is minimum, where p1 and p2 are the total number of candies in part 1 and part 2 respectively.

\subsection{Abstract}

This problem was originally designed for Dynamic Programming. However, it has been solved as a deterministic problem as it is trivial.


\subsection{Specification And Assumptions}
{\textbf {Tool Specifications:}}\\
Language used: Java\\
Platform: Ubuntu 12.04\\
Additional tools used: none\\
Eclipse Version: Version: 3.7.2\\

{\textbf{Assumptions}}\\
Input is taken one at a time\\
The list of x1 to xn is seperated by commas\\
Prints each combination of triplet whose sum is even on terminal\\


\subsection{Flow chart}
{\center{\includegraphics[height=4in]{flowchart.png}}}


\subsection{Logic Implementation}
\textbf{Following are the two main steps to solve this problem:}\\
\begin{enumerate}
\item Calculate sum of the array. If sum is odd, there can not be two subsets with equal sum, so return false.
\item If sum of array elements is even, calculate sum/2 and find a subset of array with sum equal to sum/2.
\end{enumerate}

The first step is simple. The second step is crucial, it can be solved either using recursion or Dynamic Programming.\\


\subsection{Execution Directive}
\textbf{Compiling}:

javac dp.java\\


\textbf{Running}:\\
java dp\\


\subsection{Output Of The Program}
\begin{verbatim}

Part A
Enter N
10
Enter K
4
Enter x0
10
Enter x1
20
Enter x2
30
Enter x3
3
Enter x4
3
Enter x5
3
Enter x6
3
Enter x7
3
Enter x8
3
Enter x9
2
The total number of possible triplets such that the sum is even is 64

Part B
The students get packets containing the following number of candies:
0
2	1
3	2
3	3
3	4
3	5
3	6
3	7
10	8
20	9
30	

Part C
Partition problem
true
5 = 2+3
5 = 2+3
5 = 2+3
5 = 2+3
5 = 2+3
5 = 2+3

\end{verbatim}
\subsection{Result}
Solved the trivial partition problem.

\begin{itemize}
\item Recognized when dynamic programming is a plausible approach.  E.g., recursive formulation, repeated subproblems, Global opt depends on opt subsolution, but not details thereof.
\end{itemize}
\end{document}
