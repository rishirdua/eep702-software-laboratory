
%----------------------------------------------------------------------------------------
%	PACKAGES AND OTHER DOCUMENT CONFIGURATIONS
%----------------------------------------------------------------------------------------

\documentclass[paper=a4, fontsize=11pt]{scrartcl} % A4 paper and 11pt font size

\usepackage[T1]{fontenc} % Use 8-bit encoding that has 256 glyphs
\usepackage{graphicx}
\usepackage{fourier} % Use the Adobe Utopia font for the document - comment this line to return to the LaTeX default
\usepackage[english]{babel} % English language/hyphenation
\usepackage{amsmath,amsfonts,amsthm} % Math packages

\usepackage{lipsum} % Used for inserting dummy 'Lorem ipsum' text into the template

\usepackage{sectsty} % Allows customizing section commands
\allsectionsfont{\centering \normalfont\scshape} % Make all sections centered, the default font and small caps

\usepackage{fancyhdr} % Custom headers and footers
\pagestyle{fancyplain} % Makes all pages in the document conform to the custom headers and footers
\fancyhead{} % No page header - if you want one, create it in the same way as the footers below
\fancyfoot[L]{} % Empty left footer
\fancyfoot[C]{} % Empty center footer
\fancyfoot[R]{\thepage} % Page numbering for right footer
\renewcommand{\headrulewidth}{0pt} % Remove header underlines
\renewcommand{\footrulewidth}{0pt} % Remove footer underlines
\setlength{\headheight}{13.6pt} % Customize the height of the header

\numberwithin{equation}{section} % Number equations within sections (i.e. 1.1, 1.2, 2.1, 2.2 instead of 1, 2, 3, 4)
\numberwithin{figure}{section} % Number figures within sections (i.e. 1.1, 1.2, 2.1, 2.2 instead of 1, 2, 3, 4)
\numberwithin{table}{section} % Number tables within sections (i.e. 1.1, 1.2, 2.1, 2.2 instead of 1, 2, 3, 4)

\setlength\parindent{0pt} % Removes all indentation from paragraphs - comment this line for an assignment with lots of text

%----------------------------------------------------------------------------------------
%	TITLE SECTION
%----------------------------------------------------------------------------------------

\newcommand{\horrule}[1]{\rule{\linewidth}{#1}} % Create horizontal rule command with 1 argument of height

\title{	
\normalfont \normalsize 
\textsc{EEP702, Software Laboratory} \\ [25pt] % Your university, school and/or department name(s)
\horrule{0.5pt} \\[0.4cm] % Thin top horizontal rule
\huge QT GUI Design\\ % The assignment title
\horrule{2pt} \\[0.5cm] % Thick bottom horizontal rule
}

\author{Rishi Dua, 2010EE50557} % Your name

\date{\normalsize February 20, 2014} % Today's date or a custom date

\begin{document}

\maketitle % Print the title

%----------------------------------------------------------------------------------------
%	PROBLEM 1
%----------------------------------------------------------------------------------------

\section{GUI Design in QT}


\subsection{Problem statement}
\begin {enumerate}
\item Design a user interface(GUI) in Qt. The user will be asked to enter a number in numeric
(say, 55) and will be provided a button named “Convert to text”. On pressing this button, the
entered number should be displayed in words (fifty five).

\item Add one more feature to the above GUI, which will enable user to enter a number in text
(fifty five). Add one button named “Convert to Number”, on pressing which the entered
number in words will be shown in numeric digits (55).
\end{enumerate}

\subsection{Abstract}

Qt  is a cross-platform application framework that is widely used for developing application software with a graphical user interface (GUI) (in which cases Qt is classified as a widget toolkit), and also used for developing non-GUI programs such as command-line tools and consoles for servers.
Qt uses standard C++ but makes extensive use of a special code generator (called the Meta Object Compiler, or moc) together with several macros to enrich the language. Qt can also be used in several other programming languages via language bindings. It runs on the major desktop platforms and some of the mobile platforms. It has extensive internationalization support. Non-GUI features include SQL database access, XML parsing, thread management, network support, and a unified cross-platform application programming interface (API) for file handling.\\

Qt is available under a commercial license, GPL v3 and LGPL v2. All editions support many compilers, including the GCC C++ compiler and the Visual Studio suite.\\

Qt is developed by Digia, who owns the Qt trademark, and the Qt Project under open governance, involving individual developers and firms working to advance Qt. Before the launch of the Qt Project, it was produced by Nokia's Qt Development Frameworks division, which came into existence after Nokia's acquisition of the Norwegian company Trolltech, the original producer of Qt. In February 2011, Nokia announced its decision to drop Symbian technologies and base their future smartphones on Microsoft platform instead. One month later, Nokia announced the sale of Qt's commercial licensing and professional services to Digia, with the immediate goal of taking Qt support to Android, iOS and Windows 8 platforms, and to continue focusing on desktop and embedded development, although Nokia was to remain the main development force behind the framework at that time.\\

The code is written in Qt using recursion and functions, the code is explained logically in a flow chart and in its logical implementation. the logic is mainly based on the fact that the input is to be seperated into sets of 3 numbers appropriately.\\

\newpage
\subsection{Specification And Assumptions}
{\textbf {Tool Specifications:}}\\
Qt Creat ior Version 2.4.1\\
Platform: Ubuntu 12.04\\
Bash Version: GNU bash, version 4.2.25(1)-release (x86\_64-pc-linux-gnu)\\

\textbf{Problem specifications}:
Assume the following utilities are present on the system\\
C++ compliler\\
Also works if input is negative\\


{\textbf{Assumptions}}\\
The text is printed seperating words using spaces\\

\newpage
\subsection{Flow chart}
The flowcharts is as follows

 {\center\includegraphics[scale=0.7]{flowchart.png}}\\
\newpage


\subsection{Logic Implementation}
\begin {enumerate}
\item Take input from the line box
\item If number is less than 21, print expansion from stored texts and goto step 7
\item If number is less than 100, seperate into digits and goto step 2
\item Else, seperate into 3 digits each and assign each 1000s seperator
\item For each 3 digit group, add million or billion etc. to expansion
\item Goto step 3
\item Stop
\end{enumerate}

\subsection{Execution Directive}
rightclick main.cpp and open in Qt creator\\
run the program by clicking green button\\

\subsection{Output Of The Program}
\includegraphics[scale=0.8]{ss1.png}\\

\subsection{Result}

Successfully converts a number entered into its expansion when clicked on the expand button
\textbf{Problems encountered:}
\begin{enumerate}
\item Write permissions error
\item Execute rights error
\end{enumerate}

\textbf{Solution}: check and set permissions using chmod\\
In Unix-like operating systems, chmod is the name of a Unix shell command and a system call, which both change the access permissions to file system objects (including files and directories), as well as specifying special flags.


\subsection{Conclusion}
The entered number is converted into text successfully\\
Designing in Qt is understood effectively.\
\end{document}
